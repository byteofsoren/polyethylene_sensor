% ------------- START SETTINGS -------------------------------
%\documentclass[conference]{IEEEtran}
% \documentclass[11pt,twoside,a4paper]{article}
% \usepackage{blindtext, graphicx}

%\ifCLASSINFOpdf
%\else
%\fi

% A couple of useful packages
\usepackage[table]{xcolor}
\usepackage{listings}
\usepackage{color}
\usepackage{graphics} % for pdf, bitmapped graphics files
\usepackage{epsfig} % for postscript graphics files
\usepackage{mathptmx} % assumes new font selection scheme installed
\usepackage{times} % assumes new font selection scheme installed
\usepackage{amsmath} % assumes amsmath package installed
\usepackage{amssymb}  % assumes amsmath package installed
\usepackage{mdwmath}
\usepackage{algorithm}
\usepackage{algorithmic}
\usepackage{multirow}
\usepackage[margin=0.9in]{geometry}
%\usepackage[noend]{algpseudocode}


%\addbibresource{biblatex-ela427.bib}
% Added suport for bibtext files.
% \usepackage[backend=biber, sorting=none]{biblatex}
% --- packages for drawing diagrams----
\usepackage{tikz}
\usetikzlibrary{arrows,automata}


% Package that forces figures to stay within the section.
\usepackage[section]{placeins}

%package for utf-8 support
\usepackage[utf8]{inputenc}

% added support for bloc quotes
\usepackage{csquotes}

%colour
\usepackage{color}
%\graphicspath{ {images/}{mesurment/data} }

%Captions
%\usepackage[justification=centering,font=small,labelfont=bf]{caption}
\usepackage{hyperref}
\usepackage{url}

% New over line for inverse logic
%\newcommand{\overbar}[1]{\mkern 1.5mu\overline{\mkern-1.5mu#1\mkern-1.5mu}\mkern 1.5mu}
%\setcounter{secnumdepth}{3}

% -- Panda lines needed for DataFrame to latex ---
%\newcommand{\toprule}{\hline}
%\newcommand{\midrule}{\hline}
%\newcommand{\bottomrule}{\hline}
\usepackage[pages=some,scale=1,angle=0,opacity=0.7]{background}
\newcommand\BackImage[2][scale=1]{%
%\BgThispage
\backgroundsetup{
  contents={\includegraphics[#1]{#2}}
  }
}

% Renew the indexingsystem for sections, subsections, and subsubsections so that arabic numberalsare used - Added by Ulrik
\renewcommand{\thesection}{\arabic{section}}
\renewcommand{\thesubsection}{\thesection.\arabic{subsection}}
\def\thesectiondis{\thesection.} \def\thesubsectiondis{\thesectiondis\arabic{subsection}.} \def\thesubsubsectiondis{\thesubsectiondis\arabic{subsubsection}.} \def\theparagraphdis{\thesubsubsectiondis\arabic{paragraph}.}

% Among other things, allows for items to be bold - Added by Ulrik
\usepackage{enumitem}

\usepackage{cite}
% Imported ragged for justefiing text in the \who statement
\usepackage{ragged2e}
% cite.sty was written by Donald Arseneau
\usepackage{lipsum}
\usepackage{tabularx,ragged2e,booktabs}
\newcolumntype{C}[1]{>{\Centering}m{#1}}
\renewcommand\tabularxcolumn[1]{C{#1}}
% \usepackage{multirow}
%\usepackage{longtable}
% \usepackage{supertabular,booktabs}
% V1.6 and later of IEEEtran pre-defines the format of the cite.sty package
% \cite{} output to follow that of IEEE. Loading the cite package will
% result in citation numbers being automatically sorted and properly
% "compressed/ranged". e.g., [1], [9], [2], [7], [5], [6] without using
% cite.sty will become [1], [2], [5]--[7], [9] using cite.sty. cite.sty's
% \cite will automatically add leading space, if needed. Use cite.sty's
% noadjust option (cite.sty V3.8 and later) if you want to turn this off.
% cite.sty is already installed on most LaTeX systems. Be sure and use
% version 4.0 (2003-05-27) and later if using hyperref.sty. cite.sty does
% not currently provide for hyperlinked citations.
% The latest version can be obtained at:
% http://www.ctan.org/tex-archive/macros/latex/contrib/cite/
% The documentation is contained in the cite.sty file itself.


\setcounter{tocdepth}{1}

\usepackage[justification=centering, font=small, labelfont=bf]{caption}
\captionsetup[table]{name=Table}
\newcommand{\fs}[3]{\item \textbf{#1} ($#2\text{TiB} | #3\text{TiB}$) }
\newcommand{\fsE}[3]{\item \textbf{#1} ($#2\text{TiB} | #3\text{EiB}$) }
\newcommand{\fsEE}[3]{\item \textbf{#1} ($#2\text{EiB} | #3\text{EiB}$) }
\newcommand{\fsEZ}[3]{\item \textbf{#1} ($#2\text{EiB} | #3\text{ZiB}$) }

 % Programming color don't touch without permissions.
\definecolor{codegreen}{rgb}{0,0.6,0}
\definecolor{codeblue}{HTML}{0073e6}
\definecolor{codereed}{HTML}{fc0511}
\definecolor{codegray}{rgb}{0.4,0.4,0.4}
\definecolor{codeblac}{rgb}{0.9,0.9,0.9}
\definecolor{commentgray}{rgb}{0.7,0.7,0.7}
\definecolor{codepurple}{HTML}{3ad0d8}
\definecolor{stringblue}{HTML}{0073e6}
\definecolor{stringpurple}{HTML}{ff99ff}
\definecolor{stringgray}{rgb}{0.6,0.6,0.6}
\definecolor{backcolour}{rgb}{1,1,1}
\definecolor{graybackcolour}{rgb}{0.9,0.9,0.9}

\lstdefinestyle{bashStyle}{
    backgroundcolor=\color{graybackcolour},
    commentstyle=\color{commentgray},
    keywordstyle=\color{codeblue},
    numberstyle=\tiny\color{stringblue},
    stringstyle=\color{codegreen},
    basicstyle=\normalsize,
    breakatwhitespace=false,
    breaklines=true,
    captionpos=b,
    keepspaces=true,
    numbers=left,
    numbersep=4pt,
    showspaces=false,
    showstringspaces=false,
    showtabs=false,
    tabsize=4
}

\definecolor{pythonBG}{HTML}{f7faf2}

\lstdefinestyle{pythonStyle}{
    backgroundcolor=\color{pythonBG},
    commentstyle=\color{commentgray},
    keywordstyle=\color{codereed},
    numberstyle=\tiny\color{codepurple},
    stringstyle=\color{codegreen},
    basicstyle=\footnotesize,
    breakatwhitespace=false,
    breaklines=true,
    captionpos=b,
    keepspaces=true,
    numbers=left,
    numbersep=4pt,
    showspaces=false,
    showstringspaces=false,
    showtabs=false,
    tabsize=4
}

\definecolor{codegreen}{rgb}{0,0.6,0}
\definecolor{codegray}{rgb}{0.5,0.5,0.5}
\definecolor{codepurple}{rgb}{0.58,0,0.82}
\definecolor{backcolour}{rgb}{0.95,0.95,0.92}

\lstdefinestyle{textStyle}{
    backgroundcolor=\color{backcolour},
    commentstyle=\color{codegreen},
    keywordstyle=\color{magenta},
    numberstyle=\tiny\color{codegray},
    stringstyle=\color{codepurple},
    basicstyle=\footnotesize,
    breakatwhitespace=false,
    breaklines=true,
    captionpos=b,
    keepspaces=true,
    numbers=left,
    numbersep=4pt,
    showspaces=false,
    showstringspaces=false,
    showtabs=false,
    tabsize=4
}

\definecolor{bginstructions}{HTML}{e6f7ff}

\lstdefinestyle{instructStyle}{
    backgroundcolor=\color{bginstructions},
    numberstyle=\tiny\color{codegray},
    breakatwhitespace=false,
    breaklines=true,
    captionpos=b,
    keepspaces=true,
    numbers=left,
    numbersep=4pt,
    showspaces=false,
    showstringspaces=false,
    showtabs=false,
    tabsize=4
}



%\lstset{style=mystyle}
\lstdefinestyle{bash} {language=bash,style=bashStyle}
\lstdefinestyle{python} {language=Python,style=pythonStyle, morekeywords={with, as}}
\lstdefinestyle{python_style} {language=Python,style=pythonStyle, morekeywords={with, as}}
\lstdefinestyle{gcc} {language=c,style=bashStyle}
\lstdefinestyle{text} {style=textStyle, frame=lines}
\lstdefinestyle{instruct} {style=instructStyle, frame=lines, numbers=none}



\newcommand{\fsRAM}[1]{\item \textbf{#1} (nan$|$nan) }

\newcommand{\si}[1]{$(#1)$}



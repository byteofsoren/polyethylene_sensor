\documentclass{beamer}
\input{preamble.tex}
\usepackage[utf8]{inputenc}
\title{Using High Density Conductive
Polyethylene Black Foam as a restive sensor to
build spatial object awareness in robotic grippers}
\author{Magnus Sörensen, Daniel Stenekap}
\institute{Märlardalens högskola}
\date{2019}
\begin{document}
\frame{\titlepage}


\begin{frame}
    \tableofcontents
\end{frame}

\section{Introduction}%
\label{sec:intro}

\begin{frame}
    \frametitle{Introduction}

    \begin{center}
        \includegraphics[width=.5\textwidth]{img/rob0078.jpg}
    \end{center}
    % #doc .NH
    % #doc Introduction
    % #doc .PP
    % #doc Today, the most common robot gripper still looks like the one this picture. 
    % #doc Two flat sheets, often made out of metal, press down on any tool that the arm is trying to grab.
    % #doc There is no way to get information on shape, texture or features of the object.
    % #doc In this work, we attempt to create a sensor that could begin to rectify the absence of these features.
\end{frame}

\begin{frame}
    \frametitle{High Density Conductive
Polyethylene Black Foam}
    What is High Density Conductive Polyethylene Black Foam?
    \begin{center}
        \includegraphics[width=.5\textwidth]{img/foam.jpg}
    \end{center}
    % #doc .NH
    % #doc Conductive foam.
    % #doc .PP
    % #doc What is Conductive foam (High Density Conductive Polyethylene Black Foam) you might wonder.
    % #doc This is the foam that we use to protect Integrated circuits from static electricity.
    % #doc Amoungst other features, the foam has excelent conductive ability to counteract static charges, 
    % #doc which is exactly what enables us to use it in our implementation. 
\end{frame}

\section{Method}%
\label{sec:Method}


\begin{frame}
    \frametitle{How does it work?}
    \begin{center}
        \includegraphics[width=\textwidth]{img/foam_howto.png}
    \end{center}
    % #doc .NH
    % #doc How it works - A simplified model
    % #doc .PP
    % #doc First recall that the material is conductive so the material have a given resistance per meter.
    % #doc Now, the structure of the material is full of holes which obviously counduct very poorly.
    % #doc Thus, the electrons that moves through the foam move on thr surface of bubbles inside the material.
    % #doc The final property of the conductive foam that is utilized is that when squeezed, it will return to its original* shape.
    % #doc When the material is squeezed, the bubbles are pressed togather basicaly short circuiting the surfaces that the electrons travel decreasing resistance.
    % #doc It has not been in the scoped of our project to test how durable this material is.
\end{frame}

\begin{frame}
    \frametitle{What was done}
    \begin{center}
        \includegraphics[width=.8\textwidth]{img/sensor_with_arrows_and_text.png}
    \end{center}
    % #doc .NH
    % #doc What was done
    % #doc .PP
    % #doc The approach to measure this is based an grid array of measurement pins inserted in to the bottom
    % #doc of the foam.
\end{frame}

\begin{frame}
    \frametitle{What was done}
    \begin{center}
        \includegraphics[width=.8\textwidth]{img/resistor_mapp_modell.png}
    \end{center}
    % #doc .PP
    % #doc That grid of measurement pins in the foam could be abstracted down to a resistor map as shown.
    % #doc In the implementation as shown later the pins in this map will be altering from positive to negative.
    % #doc And by the implemented algorithm a image of the features in the pressed down figure could be shown.
\end{frame}

\begin{frame}
    \frametitle{How the measurements was done}
    \begin{center}
        \includegraphics[width=.8\textwidth]{img/arduino_foam.png}
    \end{center}
    % #doc .NH
    % #doc How the measurements was done.
    % #doc .PP
    % #doc An Arduino was used for this project to measure the foam.
    % #doc In this figure an high abstraction electrical flow chart is presented.
    % #doc Two muxes is used to switch the measurements from pair to pair.
    % #doc Observe that the resitor with the variable resistance in the foam acts as
    % #doc an part of a voltage divider. Read by the
    % #doc .B Vread
    % #doc input to the Arduino.
\end{frame}

\begin{frame}
    \frametitle{The algorithm}
    \begin{center}
        \includegraphics[width=\textwidth]{img/grid0.png}
    \end{center}
    % #doc .NH
    % #doc How the algorithm works
    % #doc .PP
    % #doc To generate an matrix from the measurements the following algorithm visualised here are proposed.
    % #doc On the left the abstracted resistor grid representing the foam and on the right an outline of an matrix is shown.
    % #doc In this demo the matrix wont be filled with values. Instead the focus is to show how the values are acquired.
\end{frame}

\begin{frame}
    \frametitle{The algorithm}
    \begin{center}
        \includegraphics[width=.8\textwidth]{img/grid1.png}
    \end{center}
    % #doc .NH 2
    % #doc Measurements
    % #doc .IP 0
    % #doc The first measurement is done by taking the two highlighted resistors
\end{frame}
\begin{frame}
    \frametitle{The algorithm}
    \begin{center}
        \includegraphics[width=.8\textwidth]{img/grid2.png}
    \end{center}
    % #doc .IP 1
    % #doc Second measurement
\end{frame}
\begin{frame}
    \frametitle{The algorithm}
    \begin{center}
        \includegraphics[width=.8\textwidth]{img/grid3.png}
    \end{center}
    % #doc .IP 2
    % #doc , Third measurement
\end{frame}

\begin{frame}
    \frametitle{The algorithm}
    \begin{center}
        \includegraphics[width=.8\textwidth]{img/grid4.png}
    \end{center}
    % #doc .IP 3
    % #doc Last measurement
\end{frame}

\begin{frame}
    \frametitle{The algorithm}
    \begin{center}
        \includegraphics[width=.8\textwidth]{img/grid8.png}
    \end{center}
\end{frame}

\begin{frame}
    \frametitle{Hardware implementation}
    \begin{center}
        \begin{tabular}{ccc}
            \includegraphics[width=.33\textwidth]{img/arduino.png}&
            \includegraphics[width=.33\textwidth]{img/board.png}&
            \includegraphics[width=.33\textwidth]{img/foam_table.png}
        \end{tabular}
    \end{center}
    % #doc .NH
    % #doc Hardware implementation.
    % #doc .PP
    % #doc The hardware that was intended to be implemented in this project
    % #doc have tree parts.
    % #doc .IP 1
    % #doc On the left an standard Arduino
    % #doc .IP 2
    % #doc In the middle the board with the multiplexers are shown.
    % #doc .IP 3
    % #doc This construction is used to so that standard lab cables could be used for measurements.

\end{frame}

\begin{frame}
    \frametitle{Results}
    \begin{center}
        Due to project complexity, the electronics of the project could not be completed within time scope.
    \end{center}
    \begin{itemize}
        \item Proof of concept measurements of the foam resistance drop when exposed to pressure.
        \item Result from program with dummy text string as input is shown on the next slide. This corresponds somewhat to Proof of concept measurements.
    \end{itemize}
    % #doc .NH
    % #doc Results.
    % #doc .PP
    % #doc Do to project complexity, time and unforeseen problems like the Voltera accuracy smears out the traces and don't support Linux.
    % #doc What we did instead. Proof of concept measurements of the foam resistance drop when exposed to pressure.
    % #doc Result from program with dummy text string as input is shown on the next slide. This corresponds somewhat to Proof of concept measurements.
    % #doc We still intent to complete the project do to personal interest.

\end{frame}

\begin{frame}
    \frametitle{Results}
    \begin{center}
        \includegraphics[width=.6\textwidth]{img/foam_plot.png}
    \end{center}

\end{frame}

\begin{frame}
    \frametitle{Results}
    \begin{center}
        \includegraphics[width=.8\textwidth]{img/VoltageToMatrixMap_withTestArray.PNG}
    \end{center}

\end{frame}

\begin{frame}
    \frametitle{The algorithm}
    \begin{center}
        \includegraphics[width=\textwidth]{img/grid0.png}
    \end{center}
\end{frame}

\begin{frame}
    \begin{center}
        \textbf{The end}\\
        Questions?
    \end{center}
\end{frame}

\end{document}

